\documentclass[a4j,12pt,onecolumn,oneside,titlepage,openany,final]{jreport}
\setlength{\topmargin}{0cm}
\setlength{\oddsidemargin}{0cm}
\setlength{\textwidth}{16cm}

\usepackage[dvipdfmx]{graphicx}
\usepackage{url}
\usepackage{comment} %コメントアウト
\usepackage{here} %図の強制表示
\renewcommand{\bibname}{参考文献}
\setcounter{secnumdepth}{4}

\title{
 \Huge{Multi-CDNにおける}\\
 \Huge{リクエスト誘導技術の開発}
 \vspace{5.5cm}\\
}
\author{\LARGE{金丸 侑賢}\vspace{2cm}\\
\LARGE{九州産業大学大学院情報科学研究科}\vspace{0.5cm}\\
\LARGE{情報科学専攻}\vspace{1cm}\\
}
\date{\LARGE{平成31年1月}}
\pagestyle{plain}

\begin{document}
\maketitle
\tableofcontents
\listoffigures
\listoftables

%---------------------------------------------------------------------    1章    ---------------------------------------------------------------------
\chapter{序論}\label{chap:Prologue}
%コミットの練習

\section{研究背景}
Google Public DNS が全部悪い。
そのほかの大規模 Public DNS 群も結構悪い。
 
\section{研究目的}
本研究の目的は,Multi-CDN の環境で柔軟なリクエスト誘導を可能にすることである.
この目的を達成するために,TENBIN を ECS に対応させることで,Multi-CDN の環境でも柔軟なリクエスト誘導可能な技術を開発する.
そして、世界征服を夢見る。

\section{論文構成}
 本論文は全6章から成り,構成は以下のとおりである.
   
 \ref{chap:CDN}章では CDN ついて述べる.
 \ref{chap:Design}章では Multi-CDN におけるリクエスト誘導技術の設計について説明する.
 \ref{chap:Implement}章では Multi-CDN におけるリクエスト誘導技術の実装について述べる.
 \ref{chap:Evaluation}章では評価について述べる.
 \ref{chap:Related_Research}章では関連研究について述べる.
 最後の\ref{chap:Conclusion}章では本論文の結論を述べる.


%---------------------------------------------------------------------    2章    ---------------------------------------------------------------------
\chapter{CDN}\label{chap:CDN}
本章では,CDNのリクエスト誘導方法について説明し,Multi-CDNについて述べる.
また,DNSを用いたリクエスト誘導における問題点について説明する.

\section{リクエスト誘導方法}
本節では,CDN におけるリクエスト誘導方法としてメタファイル,エニーキャスト,DNS について述べる.
\subsection{メタファイル}
メタファイルとは,動画ファイルの場所や,再生時間,再生順序が記述されているファイルのことで,動画配信のリクエスト誘導で利用されている.
メタファイルを用いたリクエスト誘導では,2つの利点がある.
1つ目は,利用が容易である点である.
2つ目は,ファイル単位での細かいリクエスト誘導が可能である点である.
しかし,欠点として,動画配信におけるサービスしか利用することができない.
\subsection{エニーキャスト}
エニーキャストとは,ネットワーク上に存在する複数の同じIPアドレスのサーバの中から,経路上一番近いサーバを選択し通信することである.
エニキャストを用いたリクエスト誘導の利点として,全てのサービスにおいて利用が可能であるという点がある.
しかし,欠点として,エニキャストを用いたリクエスト誘導では経路情報を用いるため,経路情報を変更することは一般的に難しいことから,利用が困難である.
また,IPアドレス単位でのリクエスト誘導しかできない.
\subsection{DNS}
DNS とは,ドメイン名とIPアドレスを対応付けて管理するシステムである.
DNS を用いたリクエスト誘導では,2つの利点がある.
1つ目は,利用が容易である点である.
2つ目は,全てのサービスにおいて利用が可能である点である.
しかし,欠点として、ホスト名単位でのリクエスト誘導しかできない.
\section{Multi-CDN}
\section{問題点}

 
%---------------------------------------------------------------------    3章    ---------------------------------------------------------------------
\chapter{Multi-CDN におけるリクエスト誘導技術の設計}\label{chap:Design}
本章では,サーバ選択,選択ポリシー,TENBIN について説明する.
また,ECS,EDNS0 について説明する.
\section{サーバ選択}\label{server_select}
チケット販売サイトのサーバがダウンしてしまうというようなサービス品質の低下問題を解決するためには,広域分散型システムを用いるのが有効である.
広域分散配置した複数のサーバに対して,利用者からのアクセスをどのようにして振り分けるかが,多数のサービス要求を処理するために重要である.
また,広域分散型システムにおいて,ある特定のサーバにサービス要求が集中するとあ,複数のサーバを配置している意味が薄くなる.
そこで,広域分散型システムを構築するときには,これを構成する各サーバにサービス要求を効率的に分散しなければならない.
そのためには,複数のサーバの中から最適なサーバを選択することが必要となる.これをサーバ選択と呼ぶ.

\section{選択ポリシー}\label{select_policy}
\ref{server_select}節で述べたように,サーバ選択とは広域分散型システムを構成するサーバから最適な 1 台を選択することである.
「最適な 1 台」を決定するには,何らかの判断基準が必要である.
サーバ選択の判断基準はサービスの内容など様々な要因に依存する.
サーバ選択の指針を選択ポリシーと呼ぶ.
サーバ選択に必要となる選択ポリシーは多様であり,広域分散型システムにおける複数のサーバの中から最適な 1 台を選択する方法は,システムやサービスの性質によって様々に変化する.

\section{多様な選択ポリシーを利用可能なサーバ選択機構: TENBIN}
TENBIN は,DNS を基盤としたサーバ選択機構である.
サーバ選択を行う際,サーバ選択ポリシーと呼ぶルールを用いる.
\ref{select_policy}節で述べたように,サーバ選択の基準はサービスの内容によって異なる.
必要に応じてサーバ選択の基準を,サーバ選択ポリシーとしてプログラムし,システムに組み込むことができるのが最大の特徴である.
このサーバ選択ポリシーを記述したプログラムのことをポリシーモジュールと呼ぶ.

\section{ECS}
ECS とは,DNSの拡張プロトコルである EDNS0\cite{edns0} を用いて,問い合わせ先のDNSサーバに対して問い合わせ元の情報を伝達する技術である.

\section{EDNS0}
EDNS0 (Extension mechanism for DNS version 0) とは,DNS で扱うパケットサイズを拡張するプロトコルである.
このプロトコルを用いることで,オプションとして様々な情報を伝達ことが可能になった.
そのオプションの1つとして ECS が存在する.

%---------------------------------------------------------------------    4章    ---------------------------------------------------------------------
\chapter{Multi-CDN におけるリクエスト誘導技術の実装}\label{chap:Implement}

\section{実装環境}
\subsection{Ruby}
Ruby\cite{ruby} とは,まつもとゆきひろ氏により開発されたオブジェクト指向スクリプト言語である.
機能として,クラス定義,ガベージコレクション,強力な正規表現処理,マルチスレッド,例外処理,イテレータ・クロージャ,Mixin,演算子オーバーロード等がある.
また,ライブラリを呼び出す拡張モジュールを組み込むことが出来る.
TENBIN のプログラムは Ruby で実装されている.

\subsection{YAML}
YAML\cite{yaml} とは,構造化されたデータやコンピュータプログラムの扱うオブジェクトを文字列として表現するためのデータ形式の一つである.
ソフトウェアの設定ファイルやデー タの保存,交換などで用いられる.
TENBIN の
本研究ではユーザ設定ファイルの記述形式に YAML を用いた。


\section{ECS に対応したリクエスト誘導技術の実装}
\section{EDNS0 の実装}
\section{ECS の実装}

%---------------------------------------------------------------------    5章    ---------------------------------------------------------------------
\chapter{評価}\label{chap:Evaluation}
\section{評価概要}
\section{評価環境}
\section{評価方法}
\section{評価結果}
\section{考察}

%---------------------------------------------------------------------    6章    ---------------------------------------------------------------------
\chapter{関連研究}\label{chap:Related_Research}
\section{}

%---------------------------------------------------------------------    7章    ---------------------------------------------------------------------
\chapter{結論}\label{chap:Conclusion}
\section{まとめ}
\section{今後の課題}

\chapter*{謝辞}
本研究を進めるにあたり,ご指導を頂いた指導教員の下川俊彦教授,神屋郁子助教に感謝致します.
また,副査であるOO教授,OO教授に感謝致します。
また,共同研究を通じて多くのご意見をくださった NTT コミュニケーションズ株式会社の棚橋氏,沖本氏,亀井氏,廣瀬氏,技術開発部の皆様に感謝いたします.
また九州ギガポッププロジェクト QGPOP で本研究に対して様々なご意見をくださった,西日本電信電話株式会社 谷崎文義氏,他皆様に感謝致します.
日常の議論を通じて多くの知識や示唆を頂いた下川研究室の皆様に感謝致します.


\begin{thebibliography}{99}
\bibitem{rfc1034}
P. Mockapetris,``DOMAIN NAMES - CONCEPTS AND FACILITIES",
RFC1034,Nov. 1987.

\bibitem{rfc1035}
P. Mockapetris,``DOMAIN NAMES - IMPLEMENTATION AND SPECIFICATION",
RFC1035,Nov. 1987.

\bibitem{googlepublicdns}
``Google Public DNS",\url{https://developers.google.com/speed/public-dns}

\bibitem{opendns}
``OpenDNS",\url{https://www.opendns.com/}

\bibitem{ecs}
Carlo Contavalli, Wilmer Van Der Gaast, D. Lawrence, and W. Kumari,
``Client Subnet in DNS Queries",RFC 7871,May 2016.

\bibitem{edns0}
J. Damas, M. Graff, and P. Vixie,``Extension Mechanisms for DNS (EDNS(0))",
RFC 6891,April 2013.

\bibitem{ruby} 
``オブジェクト指向スクリプト言語Ruby", \url{https://www.ruby-lang.org/ja/}

\bibitem{yaml} 
``The Official YAML Web Site",\url{http://www.yaml.org/}


\end{thebibliography}

\appendix

\end{document}
